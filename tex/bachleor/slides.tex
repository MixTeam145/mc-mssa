\documentclass[pdf,notheorems,10pt,intlimits,unicode]{beamer}
\usetheme[numbers,totalnumbers,compress,nologo]{Statmod}
\usefonttheme[onlymath]{serif}
\setbeamertemplate{navigation symbols}{}

\usepackage[utf8]{inputenc}
\usepackage[T2A]{fontenc}
\usepackage[russian]{babel}

\usepackage{tikz}
\usepackage{ragged2e}
\usepackage{hhline}
\usepackage{array}
\usepackage[table,x11names]{xcolor}

\AtBeginSection[]{
  \begin{frame}[noframenumbering]
  \vfill
  \centering
  \begin{beamercolorbox}[sep=8pt,center,shadow=true,rounded=true]{title}
    \usebeamerfont{title}\insertsectionhead\par%
  \end{beamercolorbox}
  \vfill
  \end{frame}
}

\setbeamerfont{institute}{size=\normalsize}
\setbeamercolor{bluetext_color}{fg=blue}

\newtheorem{remark}{Замечание}
\newtheorem{algorithm}{Алгоритм}
\newtheorem{assumption}{Предположение}
\newtheorem{definition}{Определение}

\newcommand{\bfxi}{\boldsymbol{\xi}}
\newcommand{\bluetext}[1]{{\usebeamercolor[fg]{bluetext_color}#1}}
\newcommand{\redtext}[1]{{\usebeamercolor[fg]{redtext_color}#1}}

\title[Метод Монте-Карло SSA]{Метод Монте-Карло SSA для одномерных и многомерных временных рядов}

\author{Потешкин Егор Павлович, гр.20.Б04-мм}

\institute[Санкт-Петербургский Государственный Университет]{%
	\small
	Санкт-Петербургский государственный университет\\
	Прикладная математика и информатика\\
	Кафедра статистического моделирования \vspace{1.25cm}

  Научный руководитель --- д.\,ф.-м.\,н. Н.\,Э.~Голяндина\\
  Рецензент --- программист A.\,Ю.~Шлемов, Майкрософт
}

\date{Санкт-Петербург, 2024}

\input{../letters_series_mathbb.tex}

\begin{document}

\begin{frame}[plain]
  \titlepage
  \note{Научный руководитель  д.ф.-м.н., доцент Голяндина\,Н.\,Э.,\\
    кафедра статистического моделирования}
\end{frame}

\begin{frame}{Введение и постановка задачи}
  $\tX=(x_1,\ldots,x_N)$ "--- временной ряд длины $N$.\medskip

  \bluetext{Модель}: $\tX=\tT+\tH+\tR$, где $\tT$ "--- тренд, $\tH$ "--- периодическая компонента и $\tR$ "--- шум, случайная составляющая.\medskip

  \bluetext{Проблемы}:
  \begin{enumerate}
    \item Как выделить неслучайные компоненты $\tT$ и $\tH$?
    \item Как проверить наличие сигнала $\tS=\tT+\tH$?
  \end{enumerate}

  \bluetext{Методы}:
  \begin{enumerate}
    \item Singular spectrum analysis (SSA)~[Broomhead and King, 1986],\linebreak\ [Golyandina, Nekrutkin and Zhigljavsky, 2001].
    \item Monte Carlo SSA (MC-SSA)~[Allen and Smith, 1996].
  \end{enumerate}\bigskip

  MSSA и MC-MSSA "--- обобщение SSA и MC-SSA на многомерный случай.\medskip
	
  \bluetext{Задачи}:
  \begin{enumerate}
	  \item Реализовать Toeplitz MSSA и сравнить с обычным MSSA.
	  \item Сравнить модификации MC-MSSA.
    \item Рассмотреть MC-SSA в условиях реальных задач.
  \end{enumerate}
\end{frame}

\section{Глава 1. Метод MSSA}
\begin{frame}{Метод SSA и его многомерное обобщение}
  \bluetext{Входные данные}: временной ряд $\tX=(x_1,\ldots,x_N)$.\medskip
		
	\bluetext{Параметр}: длина окна $L$, $1<L<N$.\medskip
		
	\bluetext{Результат}: $m$ восстановленных составляющих временного ряда.\medskip
  \begin{enumerate}
    \item \textbf{Вложение}: $\bfX=\cT(\tX)=[X_1:\ldots:X_K]$, где $X_i=(x_1,\ldots,x_{i+L-1})^\rmT$, $K=N-L+1$.
    \item \textbf{Разложение}: $\bfX=\sum_{i=1}^r \sigma_i P_i Q_i^\rmT=\bfX_1+\ldots+\bfX_r$, $\operatorname{rank}\bfX_i=1$.
    \item \textbf{Группировка}: $\bfX=\bfX_{I_1}+\ldots+\bfX_{I_m}$, где $\bfX_{I_k}=\sum_{i\in I_k}\bfX_i$.
    \item \textbf{Восстановление}: $\tX=\widetilde\tX_{I_1}+\ldots+\widetilde\tX_{I_m}$, где $\widetilde\tX_{I_k}=\cT^{-1}\circ\cH(\bfX_{I_k})$.
  \end{enumerate}\bigskip

  \bluetext{MSSA}: $\tX^{(1)},\ldots,\tX^{(D)}$ "--- временные ряды.\medskip

  Составим $\tX=\{\tX^{(d)}\}_{d=1}^D$ "--- $D$-канальный временной ряд.\medskip

  Тогда на шаге вложения $\bfX=[\bfX^{(1)}:\ldots:\bfX^{(D)}]$, где $\bfX^{(i)}=\cT(\tX^{(i)})$.
\end{frame}

\begin{frame}{Модификации MSSA}
  Модификации MSSA отличаются только шагом разложения.\medskip

  \bluetext{Basic MSSA}: сингулярное разложение (SVD) $\bfX$.\medskip
		
	\bluetext{Toeplitz Block MSSA}~[Plaut and Vautard, 1994]: $Q_i$ "---  ортонормированные собственные векторы матрицы
  \[
    \bfT_{\text{Block}}=
    \begin{pmatrix}
    \bfT^{(K)}_{1,1} & \bfT^{(K)}_{1,2} & \cdots & \bfT^{(K)}_{1,D} \\
    \bfT^{(K)}_{2,1} & \bfT^{(K)}_{2,2} & \cdots & \bfT^{(K)}_{2,D} \\
    \vdots  & \vdots  & \ddots & \vdots  \\
    \bfT^{(K)}_{D,1} & \bfT^{(K)}_{D,D} & \cdots & \bfT^{(K)}_{D,D}
    \end{pmatrix}\in \mathbb{R}^{DK\times DK},
  \]
  где $\bfT^{(K)}_{l,k}$ "--- матрица с элементами
  \begin{equation*}
    \left(\bfT^{(K)}_{l,k}\right)_{ij}=\frac{1}{N-|i-j|}\sum_{n=1}^{N-|i-j|} x^{(l)}_nx^{(k)}_{n+|i-j|},\ 1\leqslant i,j\leqslant K.
  \end{equation*}

  \bluetext{\textbf{Toeplitz Sum MSSA}}: $P_i$ "--- ортонормированные собственные векторы матрицы $\bfT_{\text{Sum}}=\sum_{i=1}^D \bfT^{(L)}_{i,i}\in\mathbb{R}^{L\times L}$, предлагается в этой работе.
\end{frame}

\begin{frame}{Численное исследование}
  \bluetext{Дано}: $\{\tF^{(1)}, \tF^{(2)}\}=\{\tS^{(1)},\tS^{(2)}\} + \{\tR^{(1)},\tR^{(2)}\}$, где $\tR$ "--- независимые реализации белого гауссовского шума с $\sigma^2=25$, $N=71$.\medskip

  \bluetext{Задача}: проверить точность базового и модифицированных методов для разных значений параметра $L$. \medskip

  \bluetext{Рассмотрим 3 случая}:\medskip
  \begin{enumerate}
    \item Косинусы с одинаковыми частотами:
    \[
    s_n^{(1)}=30\cos(2\pi n/12),\quad s_n^{(2)}=20\cos(2\pi n/12),\quad n=1,\ldots, N.
    \]
    \item Косинусы с разными частотами:
    \[
    s_n^{(1)}=30\cos(2\pi n/12),\quad s_n^{(2)}=20\cos(2\pi n/8),\quad n=1,\ldots, N.
    \]
    \item Полиномы первой степени (нестационарные ряды):
    \[
    s_n^{(1)}=1.2n,\quad s_n^{(2)}=0.8n,\quad n=1,\ldots,N.
    \]
  \end{enumerate}
\end{frame}

\begin{frame}{Численное исследование. Результаты}
  \begin{table}[h]
    \centering
    \caption{MSE восстановления сигнала.}
    \scalebox{0.8}{
      \begin{tabular}{cccccc}\hline
        & $L=12$ & $L=24$ & $L=36$ & $L=48$ & $L=60$\\
        & $DK=120$ & $DK=96$ & $DK=72$ & $DK=48$ & $DK=24$\\
        \hhline{======}
        \rowcolor{lightgray} Случай 1 ($\omega_1=\omega_2$) & & & & &\\
        \hline
        MSSA & $3.18$ & $1.83$ & $1.59$ &  \bluetext{\textbf{1.47}} & $2.00$\\
        \hline
        Toeplitz Sum MSSA &  $3.17$ & $1.75$ & $1.44$ & \bluetext{\textbf{1.32}} & \bluetext{\textbf{1.33}}\\
        \hline
        Toeplitz Block MSSA & $1.39$ & \textcolor{red}{\textbf{1.26}} & \textcolor{red}{\textbf{1.25}} & $1.33$ & $1.97$\\
        \hhline{======}
        \rowcolor{lightgray} Случай 2 ($\omega_1\ne\omega_2$) & & & & &\\
        \hline
        MSSA & $6.91$ & $3.77$ & $3.07$ & \bluetext{\textbf{2.88}} & $3.84$\\
        \hline
        Toeplitz Sum MSSA & $6.88$ & $3.65$ & $2.64$ & $2.37$ & \textcolor{red}{\textbf{2.27}}\\
        \hline
        Toeplitz Block MSSA & $4.47$ & $3.67$ & \bluetext{\textbf{3.22}} & \bluetext{\textbf{3.23}} & $3.8$\\
        \hhline{======}
        \rowcolor{lightgray} Случай 3 (тренд) & & & & &\\
        \hline
        MSSA & $3.42$ & $1.94$ & $1.63$ & \textcolor{red}{\textbf{1.57}} & $2.27$\\
        \hline
        Toeplitz Sum MSSA & $3.32$  & \bluetext{\textbf{2.24}} & $3.04$ & $5.91$ & $11.95$ \\
        \hline
        Toeplitz Block MSSA & $12.55$ & $6.18$  & $2.97$ & \bluetext{\textbf{1.78}} & $1.97$\\
        \hline
    \end{tabular}}
    \label{tab:mse}
  \end{table}
  \bluetext{Трудоемкость}: нахождение собственных векторов. Toeplitz Block MSSA зависит от $DK$, а Toeplitz Sum MSSA "--- от $L$.
\end{frame}

\section{Глава 2. Метод Monte Carlo (M)SSA}
\begin{frame}{Статистический критерий с суррогатными выборками}
  Если распределение статистики $T$ при верной $H_0$ неизвестно, оно оценивается с помощью $G$ суррогатных выборок:
  \begin{enumerate}
    \item По выборке $X=(X_1,\ldots, X_n)$ строится статистика критерия $t=T(X)$.
    \item Моделируется $G$ выборок $R_1,\ldots, R_G$ объема $n$ при верной $H_0$, и строятся статистики $t_i=T(R_i)$, $i=1,\ldots, G$.
		\item Находится оценка $t_\alpha$ критического значения $\hat t_\alpha$ как выборочный $(1-\alpha)$-квантиль $(t_1,\ldots,t_G)$.
    \item Если $t>\hat t_\alpha$, то $H_0$ отвергается.
  \end{enumerate}

  \begin{remark}
    Оценка критического значения $t_\alpha$ по квантилям выборки объема $G$ имеет смысл при $\alpha>1/G$, так как минимальное и максимальное значения выборки соответствуют $\alpha$- и $(1-\alpha)$-квантилям при $\alpha = 1/G$. Если $\alpha < 1/G$, критерий с суррогатными выборками радикальнее исходного критерия.
  \end{remark}
\end{frame}

\begin{frame}{Поправка неточных критериев}
  Приведем алгоритм, преобразовывающий неточный критерий (консервативный или радикальный) в асимптотически точный.\medskip

  Зафиксируем $H_0$, уровень значимости $\alpha^*$, количество выборок $M$ для оценки $\alpha_I(\alpha)$ и их объем $N$:
  \begin{enumerate}
    \item Моделируется $M$ выборок объема $N$ при верной $H_0$.
    \item По моделированным данным строится зависимость ошибки первого рода от уровня значимости $\alpha_I(\alpha)$.
    \item Рассчитывается формальный уровень значимости: $\widetilde{\alpha}^*=\alpha_I^{-1}(\alpha^*)$. Критерий с таким уровнем значимости является асимптотически точным при $M\to\infty$.
   \end{enumerate}

  \begin{remark}
    Критерий с суррогатными выборками, к которому применяется поправка, не должен быть слишком радикальным. Для сильно радикальных критериев с очень маленьким $\widetilde\alpha^*$ условие $G > 1/\widetilde\alpha^*$ приводит к очень большим вычислительным затратам и, тем самым, к практической нереализуемости.
  \end{remark}
\end{frame}

\begin{frame}{Multiple Monte Carlo SSA}

  \bluetext{Модель}: $\tX=\tS + \boldsymbol{\xi}$, где $\tS$ "--- сигнал (колебания с какой-то частотой), $\boldsymbol{\xi}$ "--- стационарный процесс с нулевым средним.\medskip

  \bluetext{Задача}: проверить $H_0:\tS=0$ "--- отсутствие сигнала.\medskip

  \bluetext{Метод}: Multiple Monte Carlo SSA~[Golyandina, 2023]:\medskip

  \bluetext{Входные данные}: временной ряд $\tX$.

  \bluetext{Параметры}: длина окна $L$, выбор векторов $W_1,\ldots,W_H\in \mathbb{R}^{L}$, количество суррогатных реализаций $G$.
  \begin{enumerate}
    \item Строятся статистики критерия $\widehat p_k=\|\bfX^\rmT W_k\|^2$, $k=1,\ldots,H$.
    \item Оцениваются их распределения на основе $G$ суррогатных реализаций $\bfxi$.
    \item Строятся предсказательные интервалы с поправкой на множественные сравнения с уровнем доверия $(1-\alpha)$.
  \end{enumerate}

  Далее предполагаем, что $\boldsymbol{\xi}$ "--- красный шум, причем с известными параметрами.
\end{frame}

\begin{frame}{Используемый варинт MC-SSA}
  \begin{definition}
    Пусть $\bfX=\sum_i \sigma_i P_i Q_i^T$~--- любое разложение $\bfX$ в сумму матриц единичного ранга. Будем называть $P_i$ \emph{левыми}, а $Q_i$~--- \emph{правыми векторами} матрицы $\bfX$.
  \end{definition}

  В качестве векторов для проекции будем брать левые векторы матрицы $\bfX$. Варианты разложения траекторной матрицы: сингулярное, теплицево.\medskip

  \bluetext{Плюс}: если $H_0$ отверглась, можно восстановить сигнал с помощью SSA на основе значимых $W_i$.\medskip

  \bluetext{Минус}: этот вариант дает радикальный критерий, но Toeplitz MC-SSA менее радикален, что дает возможность использовать поправку неточных критериев~[Ларин, 2022].
\end{frame}

\begin{frame}{Численное сравнение MC-SSA с другими критериями}
  \bluetext{Подход}: отбелить красный шум, т.~е. сделать его белым, и затем применить статистический критерий, проверяющий гипотезу, что ряд является реализацией белого шума.\medskip

  Для проверки исходного ряда на белый шум возьмем Q-тест Бокса-Пирса~[Box and Pierce, 1970] и тест с использованием вейвлетов~[Nason and Savchev, 2014].\medskip

  \bluetext{Особенность второго теста}: длина ряда должна быть степенью двойки.\medskip

  Пусть $N=128$. Сравним такой метод с MC-SSA при разных частотах в альтернативе.\medskip

  \bluetext{Модель}: $\tX=\tS+\boldsymbol{\xi}$, где $\tS=\{A \cos(2\pi\omega n)\}_{n=1}^N$, $\boldsymbol{\xi}$ "--- красный шум с параметрами $\varphi=0.7$ и $\delta=1$.\medskip

  $H_0:A=0$, $H_1:A\ne0$.
\end{frame}

\begin{frame}{Результат численного сравнения MC-SSA с другими критериями}
  \begin{table}[h!]
    \centering
    \caption{Результаты численного сравнения MC-SSA с другими критериями ($\alpha^*=0.1$)}
    \label{tab:comparison}
    \scalebox{0.92}{
      \begin{tabular}{|cc>{\centering\arraybackslash}m{0.7in}>{\centering\arraybackslash}m{0.7in}>{\centering\arraybackslash}m{0.7in}|}\hline
        Метод & $\alpha_I(\alpha^*)$ & $\beta(\widetilde\alpha^*)$ $A=1.5$ $\omega=0.025$ & $\beta(\widetilde\alpha^*)$ $A=0.8$ $\omega=0.125$ & $\beta(\widetilde\alpha^*)$ $A=0.5$ $\omega=0.225$\\
        \hline
        MC-SSA ($L=10$) & 0.101 & 0.57 & 0.51 & 0.465\\
        \hline
        MC-SSA ($L=32$) & 0.163 & 0.566 & 0.678 & 0.668\\
        \hline
        MC-SSA ($L=64$) & 0.25 & 0.556 & 0.684 & 0.665\\
        \hline
        MC-SSA ($L=96$) & 0.593 & 0.599 & 0.734 & 0.709\\
        \hline
        MC-SSA ($L=115$) & 0.668 & \textcolor{red}{\textbf{0.668}} & \textcolor{red}{\textbf{0.791}} & \textcolor{red}{\textbf{0.753}}\\
        \hline
        box & 0.103 & 0.289 & 0.269 & 0.064\\
        \hline
        wavelet & 0.091 & 0.354 & 0.414 & 0.57\\
        \hline
      \end{tabular}
      }
    \end{table}
    \bluetext{Вывод}: MC-SSA намного мощнее box и wavelet, особенно при малых частотах сигнала.
\end{frame}

\begin{frame}{MC-SSA: обобщение на многомерный случай}
  \bluetext{MC-MSSA}: SSA заменяется на MSSA и красный шум генерируется с тем же количеством каналов, что и у исходного временного ряда.\medskip

  \begin{remark}
    В одномерном случае левые векторы матрицы $\bfX$ становятся правыми заменой $L\mapsto N - L + 1$, поэтому по-отдельности рассматривать в качестве векторов для проекции правые векторы не нужно. В многомерном случае левые и правые векторы дают \textbf{разные} критерии.
  \end{remark}

  \begin{remark}
    Для рассмотрения правых векторов матрицы $\bfX$ в качестве $W_k$ требуется заменить в алгоритме MC-SSA $\bfX$ на $\bfX^\rmT$ и $\mathbf{\Xi}$ на $\mathbf{\Xi}^\rmT$.
  \end{remark}
\end{frame}

\begin{frame}{Численное сравнение модификаций MC-MSSA}
	\bluetext{Модель}: $\tX=\tS+\bfxi$, где $\bfxi$ "--- красный шум с параметрами $\varphi$ и $\delta=1$, а $\tS$ "--- сигнал с
	\[
	s_{n}^{(1)}=s_n^{(2)}=A\cos(2\pi n\omega),\quad n=1,\ldots, N,\quad N=100.
	\]

	$H_0$: $A=0$, $H_1: A\ne0$.\medskip
	
	\bluetext{Задача}: сравнить критерии Basic MC-MSSA и Toeplitz MC-MSSA в двух вариациях.\bigskip

  Рассмотрим следующие примеры:
  \begin{enumerate}
    \item $\varphi=0.7$, $\omega=0.075$;
    \item $\varphi=0.3$, $\omega=0.075$;
    \item $\varphi=0.7$, $\omega=0.225$.
  \end{enumerate}
\end{frame}

\begin{frame}{Результат численного сравнения модификаций MC-MSSA}
  \begin{table}[h]
    \caption{Мощность методов для оптимальных $L$ при $\alpha^*=0.1$}
	  \label{tab:res_mc-ssa_power}
	  \centering
    \scalebox{0.75}{
      \begin{tabular}{|c>{\centering\arraybackslash}m{0.8in}>{\centering\arraybackslash}m{0.8in}>{\centering\arraybackslash}m{0.8in}>{\centering\arraybackslash}m{0.8in}|}\hline
        Метод & левые/правые векторы & $\beta(\widetilde\alpha^*)$ (пример 1) & $\beta(\widetilde\alpha^*)$ (пример 2) & $\beta(\widetilde\alpha^*)$ (пример 3)\\
		    \hline
		    SVD & левые & 0.754 & \textcolor{blue}{\textbf{0.399}} & 0.573 \\
		    \hline
		    SVD & правые & 0.754 & 0.382 & 0.442 \\
		    \hline
		    Block & левые & \textcolor{blue}{\textbf{0.796}} & \textcolor{blue}{\textbf{0.398}} & \textcolor{blue}{\textbf{0.597}} \\
		    \hline
		    Block & правые & 0.717 & 0.389 & 0.473 \\
		    \hline
		    Sum & левые & \textcolor{red}{\textbf{0.806}} & \textcolor{red}{\textbf{0.421}} & \textcolor{red}{\textbf{0.625}} \\
		    \hline
		    Sum & правые & 0.748 & \textcolor{blue}{\textbf{0.412}} & \textcolor{blue}{\textbf{0.613}} \\
		    \hline
      \end{tabular}
      }
    \end{table}\vspace{-1em}

    \begin{table}[h]
      \caption{Размеры матриц методов}
      \label{tab:res_mc-ssa_complexity}
      \centering
      \scalebox{0.75}{
        \begin{tabular}{|c>{\centering\arraybackslash}m{0.8in}>{\centering\arraybackslash}m{0.8in}>{\centering\arraybackslash}m{0.8in}>{\centering\arraybackslash}m{0.8in}|}\hline
          Метод & левые/правые векторы & Размер матрицы (пример 1) & Размер матрицы (пример 2) & Размер матрицы (пример 3) \\
          \hline
          SVD & левые & 50 & 10 & 20 \\
          \hline
          SVD & правые & 80 & 80 & 80 \\
          \hline
          \rowcolor{cyan} Block & левые & \textbf{102} & \textbf{162} & \textbf{162} \\
          \hline
          Block & правые & 42 & 22 & 42\\
          \hline
          \rowcolor{cyan} Sum & левые & \textbf{80} & \textbf{20} & \textbf{80} \\
          \hline
          Sum & правые & 80 & 80 & 80 \\
          \hline
        \end{tabular}
      }
    \end{table}
\end{frame}

\section{Глава 3. Применение метода Monte Carlo SSA на практике}
\begin{frame}{Зависимость радикальности и мощности MC-SSA от
  параметра $L$}
  Было произведено исследование зависимости мощности MC-SSA от длины окна $L$.\medskip

  Численные эксперименты показали, что длина окна $L$, дающая наибольшую мощность, зависит от параметров шума $\boldsymbol{\xi}$, длины ряда $N$ и \textbf{частоты сигнала} $\omega$ в $H_1$. Также с ростом $N$ радикальность критерия при фиксированном $L$ едва заметно уменьшается.\medskip

  На их основе были выработаны следующие рекомендации:
  \begin{enumerate}
    \item Без поправки использовать MC-SSA можно только с $L\approx10$.
    \item Можно построить зависимость оптимальной длины окна от параметров ряда с помощью численного моделирования. Но это возможно, если есть дополнительная информация о диапазоне возможных частот в ряде.
  \end{enumerate}
\end{frame}

\begin{frame}{MC-SSA с мешающим сигналом}
  Мешающий сигнал~--- это сигнал, который не нужно обнаруживать при проверки гипотезы о наличии сигнала (например, тренд, сезонность).\medskip

  \bluetext{Подход}: устранить влияние мешающего сигнала на векторы, на которые делается проекция.\medskip

  \bluetext{Модель}: $\tX=\tF + \tS + \boldsymbol{\xi}$, где $\tF$ "--- мешающий сигнал, $\tS$ "--- неизвестный сигнал и $\boldsymbol{\xi}$ "--- красный шум.\medskip

  Тогда $H_0:\tS=0$ и $H_1:\tS\ne 0$.\bigskip

  Получено, что мощность критерия понижается, но больше от оценки параметров шума, чем от мешающего сигнала, причем чем меньше параметр $\varphi$, тем больше потеря в мощности.\medskip
\end{frame}

\begin{frame}{Анализ реального примера}
  \begin{figure}
    \centering
    \includegraphics[width=\textwidth]{img/Nino_reconstruct_trend_slides.pdf}\vspace{-2.8em}
    \includegraphics[width=\textwidth]{img/Nino_reconstruct_season_slides.pdf}\vspace{-1em}
    \caption{Выделенный тренд ($L=120$) и годовая периодичность ($L=444$) с помощью SSA (данные по месяцам)}
  \end{figure}
\end{frame}

\begin{frame}{Анализ реального примера}
  \begin{figure}[h!]
    \centering
    \includegraphics[width=0.8\textwidth]{img/Nino_mcssa.pdf}
    \caption{Результат работы MC-SSA ($\alpha=0.05$)}
    \label{Nino_mcssa}
  \end{figure}
  Значимыми являются четыре компоненты, две компоненты, имеющие \textbf{период} приблизительно $\mathbf{6}$, легко интерпретируются~---  это замеченная \textbf{полугодовая периодичность}.
\end{frame}

\section{}
\begin{frame}{Заключение}
  Мои результаты:
	\begin{enumerate}
		\item Был реализован метод Toeplitz MSSA в вариантах Block и Sum (на языке \textsf{R}). На основе численных исследований рекомендуется использовать вариант Sum в виду численной эффективности и реализации, подходящей под структуру пакета \textsf{Rssa}.
		\item Был выработан подход к сравнению критериев, построенных на основе радикальных и использующих суррогатные выборки: выбирать наиболее мощный критерий среди не слишком радикальных.
		\item На основе этого подхода по результатам численных исследований рекомендуется использовать модификацию Toeplitz Sum MC-MSSA с проекцией на левые векторы.
    \item Проведены численные исследования по выбору оптимальной длины окна, степени искажения критерия при оценке параметров красного шума, а также случая мешающего сигнала.
    \item Показано, что метод MC-SSA намного мощнее двух рассмотренных критериев в задаче обнаружения сигнала в красном шуме, особенно при малых частотах сигнала в альтернативе.
  \end{enumerate}
\end{frame}

\end{document} 