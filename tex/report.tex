\documentclass[specialist,
substylefile = spbu_report.rtx,
subf,href,colorlinks=true, 12pt]{disser}
\usepackage[utf8]{inputenc}
\usepackage[english,russian]{babel}

\usepackage[a4paper,
mag=1000, includefoot,
left=3cm, right=1.5cm, top=2cm, bottom=2cm, headsep=1cm, footskip=1cm]{geometry}

\usepackage{graphicx,subcaption,ragged2e}

\usepackage{amsthm}
\usepackage{amsmath}
\usepackage{amssymb}

\usepackage{hhline}

\usepackage{xcolor}

\usepackage{array}

\newcommand{\traj}{\mathbf{X}}
\newcommand{\toeplitz}{\widetilde{\mathbf{C}}}
\newcommand{\transponse}{^\mathrm{T}}

\theoremstyle{definition}
\newtheorem{definition}{Определение}
\newtheorem{algorithm}{Алгоритм}
\newtheorem{remark}{Замечание}

\newcommand{\R}{\mathbb{R}}

\newcommand{\bfxi}{\boldsymbol{\xi}}

\include{letters_series_mathbb.tex}

\begin{document}
%
% Титульный лист на русском языке
%
% Название организации
\institution{%
	Санкт-Петербургский государственный университет\\
	Прикладная математика и информатика
}

\title{Отчет по учебной практике 4 (научно-исследовательской работе) (семестр 7)}

% Тема
\topic{Метод Монте-Карло SSA для многомерных временных рядов}

% Автор
\author{Потешкин Егор Павлович}
\group{группа 20.Б04-мм}

% Научный руководитель
\sa       {Голяндина Нина Эдуардовна \\%
	Кафедра Статистического Моделирования}
\sastatus {д.\,ф.-м.\,н., доцент}

% Город и год
\city{Санкт-Петербург}
\date{\number\year}

\maketitle
\tableofcontents
\intro
Метод Singular Spectrum Analysis (SSA) является мощным инструментом для анализа временных рядов. Он позволяет разложить ряд на интерпретируемые компоненты, такие как тренд, периодические колебания и шум, что значительно упрощает процесс анализа. Метод Monte-Carlo SSA, в свою очередь, решает задачу обнаружения сигнала в шуме~\cite{Golyandina_2023}.

Однако, вариант Monte-Carlo SSA для анализа многомерных временных рядов мало исследован. В работе~\cite{Larin_2022} рассматривается применение метода Monte Carlo SSA для анализа многомерных временных рядов, и авторы сталкиваются с проблемой отсутствия реализации Тёплицева варианта MSSA в пакете Rssa~\cite{Rssa}.

В этой работе была поставлена задача реализовать двумя способами метод Toeplitz MSSA, сравнить их между собой и с обычным MSSA как для методов оценки сигнала, так и для использования в Monte-Carlo MSSA, а также рассмотреть Monte-Carlo SSA в условиях реальных задач.

В главе~\ref{chpt:mssa} приведено описание метода MSSA и двух его модификаций, и их численное сравнение. В главе~\ref{chpt:mc-ssa} представлен метод Monte-Carlo SSA, и приведено численное сравнение метода с разными параметрами. В ходе работы в этом семестре была добавлена глава~\ref{chpt:mc_ssa_real}, в которой рассмотрено два способа оценки неизвестных параметров красного шума и их сравнение, а также разобран случай Monte-Carlo SSA, когда во временном ряде присутствует мешающий сигнал. 

%Рассмотрим вещественнозначный временной ряд, т.е. последовательность вещественнозначных чисел, упорядоченных по времени. Метод Singular Spectrum Analysis (SSA) позволяет представить такой ряд в виде суммы интерпретируемых компонент, таких как тренд, периодические компоненты и шум, что упрощает процесс анализа временного ряда.

%В повседневной жизни чаще всего временные ряды встречаются в виде детерминированного сигнала и случайного шума. Тогда возникает задача обнаружения сигнала в шуме. Метод Monte-Carlo SSA "--- решение этой задачи.  

%Существует вариант Monte-Carlo SSA для анализа многомерных временных рядов, однако этот вариант мало исследован. В работе~\cite{Larin_2022} были численно исследованы свойства метода Monte-Carlo SSA, однако исследование не было завершено полностью, возникла проблема необходимости реализации Тёплицева варианта Multivariate SSA (MSSA), но ее не было в используемом пакете Rssa~\cite{Rssa}.

%Целью данной работы является реализация метода Тёплицева MSSA двумя способами, а также сравнение способов между собой и с базовой версией MSSA. Было произведено сравнение точности методов в восстановлении сигнала и в использовании в Monte-Carlo MSSA.  

\chapter{Метод MSSA и его модификации}\label{chpt:mssa}
Метод Multivariate Singular Spectrum Analysis (сокращенно MSSA) состоит из четырех этапов: \emph{вложения}, \emph{разложения}, \emph{группировки} и \emph{диагонального усреднения}.
Давайте начнем с общих частей всех версий алгоритмов SSA. Этими общими частями являются процедура вложения и диагонального усреднения (ганкелизации).
\begin{definition}
	Пусть $\tX$ "--- одномерный временной ряд длины $N$. Выберем параметр $L$, называемый \emph{длиной окна}, $1<L<N$. Рассмотрим $K=N-L+1$ векторов вложения $X_i=(x_{i},\ldots, x_{i+L-1})^\rmT,\ 1\leqslant j \leqslant K$. Определим оператор вложения $\cT$ следующим образом:
	\begin{equation}\label{eq:embedding}
		\cT(\tX)=\bfX=[X_1:\ldots:X_K]=
		\begin{pmatrix}
			x_1    & x_2     & \cdots & x_K     \\
			x_2    & x_3     & \cdots & x_{K+1} \\
			\vdots & \vdots  & \ddots & \vdots  \\
			x_L    & x_{L+1} & \cdots & x_N
		\end{pmatrix}.
	\end{equation}
\end{definition}
\begin{definition}
	Матрицу $\bfX$ из~\eqref{eq:embedding} называют траекторной матрицей.
\end{definition}\noindent
Заметим, что матрица $\bfX$ является \emph{ганкелевой}, т.е на всех ее побочных диагоналях стоят одинаковые элементы, а оператор $\cT$ задает взаимно-однозначное соответствие между множеством временных рядов длины $N$ и множеством ганкелевых матриц $L\times K$.
\begin{definition}
	Пусть $\bfY=\{y_{ij}\}_{i,j=1}^{L,K}$ "--- некоторая матрица. Определим оператор ганкелизации $\cH$:
	\begin{equation}\label{eq:averaging}
		(\cH(\bfY))_{ij}=\sum_{(l,k)\in A_s}y_{lk}/w_s,
	\end{equation}
	где $s=i+j-1$, $A_s=\{(l,k)\, :\, l+k=s+1,\, 1\leqslant l\leqslant L,\, 1\leqslant k\leqslant K\}$ и $w_s=|A_s|$ "--- количество элементов в множестве $A_s$. Это соответствует
	усреднению элементов матрицы $\bfY$ по побочным диагоналям.
\end{definition}
\section{Метод MSSA}
Рассмотрим вещественнозначные одномерные временные ряды $\tX^{(d)}=(x_1^{(d)}, x_2^{(d)},\ldots, x_{N_d}^{(d)})$ длины $N_d>2$, $d=1,\ldots,D$. Составим из этих рядов $\tX=\{\tX^{(d)}\}_{d=1}^D$ "--- $D$-канальный временной ряд с длинами $N_d$.
%\begin{definition}
%	\textbf{\emph{Оператор вложения}} $\cT$ "--- линейное отображение, которое преобразует %временной ряд $\tX$ в траекторную матрицу $\bfX$.
%\end{definition}
\subsection{Вложение}\label{sect:embedding}
Зафиксируем $L$, $1<L<\min(N_1,\ldots,N_D)$. Для каждого ряда $\tX^{(d)}$ составим траекторную матрицу $\bfX^{(d)}$ . Обозначим $K=\sum_{d=1}^D K_d$. Результатом этапа вложения является траекторная матрица многоканального временного ряда
\begin{equation}\label{eq:embedding_mssa}
	\bfX=\cT_{\text{MSSA}}(\tX)=[\cT(\tX^{(1)}):\ldots:\cT(\tX^{(D)})]=[\traj^{(1)}:\ldots:\traj^{(D)}].
\end{equation}
\subsection{Разложение}
Задача этапа разложения "--- разбить траекторную матрицу $\traj$ в сумму матриц ранга 1. В базовой версии MSSA используется сингулярное разложение (SVD).

Положим $\mathbf{S}=\traj\traj\transponse$. Пусть $\lambda_i$ "--- собственные числа, а $U_i$ "--- ортонормированная система векторов матрицы $\mathbf{S}$. Упорядочим $\lambda_i$ по убыванию и найдем $p$ такое, что $\lambda_p>0$, а $\lambda_{p+1}=0$. Тогда
\begin{equation}\label{eq:svd}
	\traj=\sum_{i=1}^p\sqrt{\lambda_i}U_iV_i\transponse=\sum_{i=1}^p\traj_i,
\end{equation}
где $V_i=\traj\transponse U_i/\sqrt{\lambda_i}$. Тройку $(\sqrt{\lambda_i}, U_i, V_i)$ принято называть $i$-й собственной тройкой сингулярного разложения, $\sqrt{\lambda_i}$ "--- сингулярным числом, $U_i$ "--- левым сингулярным вектором, а $V_i$ "--- правым сингулярным вектором. Отметим, что левые сингулярные векторы имеют размерность $L$, а правые сингулярные вектора "--- размерность $K$.
\subsection{Группировка}\label{sect:grouping}
На этом шаге множество индексов $I=\{1,\ldots,p\}$ разбивается на $m$ непересекающихся множеств $I_m,\ldots,I_m$ и матрица $\traj$ представляется в виде суммы
\[
	\traj = \sum_{k=1}^m \traj_{I_k},
\]
где $\traj_{I_k}=\sum_{i\in I_k}\traj_i$.
\subsection{Диагональное усреднение}\label{sect:averaging}
Пусть $\bfY=[\bfY^{(1)}:\ldots:\bfY^{(M)}]$ "--- некоторая составная матрица, тогда оператор ганкелизации для составной матрицы
\begin{equation}\label{eq:averaging_mssa}
\cH_{\text{stacked}}(\bfY)=[\cH(\bfY^{(1)}):\ldots:\cH(\bfY^{(M)})].
\end{equation}

Финальным шагом MSSA является преобразование каждой матрицы $\traj_{I_k}$, составленной в разделе~\ref{sect:grouping}, в $D$-канальный временной ряд:
\begin{equation}
	\widetilde\tX_{I_k}=\cT_{\text{MSSA}}^{-1}\circ\cH_{\text{stacked}}\left(\bfX_{I_k}\right),
\end{equation}
где $\cT_{\text{MSSA}}$ "--- оператор вложения~\eqref{eq:embedding_mssa}, $\cH_{\text{stacked}}$ "--- оператор ганкелизации~\eqref{eq:averaging_mssa}.

%Пусть $\mathbf Y=(y_{ij})$ "--- матрица размера $L\times K$. Положим $L^*=\min(L,K)$, %$K^*=\max(L,K)$ и $N=L+K-1$. Пусть $y^*_{ij}=y_{ij}$, если $L<K$, и $y^*_{ij}=y_{ji}$ иначе. %\textit{Диагональное усреднение} переводит матрицу $\mathbf{Y}$ в ряд $g_1,\ldots,g_N$ по %формуле
%\[
%g_k=
%\begin{cases}
%	{\displaystyle\frac{1}{k}\sum_{m=1}^{k} y^*_{m,k-m+1}},&\text{при }1\leqslant k<L^*\\
%	{\displaystyle\frac{1}{L^*}\sum_{m=1}^{L^*} y^*_{m,k-m+1}},&\text{при }L^*\leqslant %k\leqslant K^* \\
%	{\displaystyle\frac{1}{N-k+1}\sum_{m=k-K^*+1}^{N-K^*+1}y^*_{m,k-m+1}},&\text{при }K^*< %k\leqslant N
%\end{cases}
%\]

%Из~\eqref{eq:traj} следует, что $\traj_{I_k}$ можно представить в следующем виде:
%\[
%\traj_{I_k}=[\traj^{(1)}_{I_k}:\ldots:\traj^{(D)}_{I_k}].
%\]
%Тогда, чтобы получить $D$-канальный временной ряд, применим диагональное усреднение к каждой %матрице $\traj_{I_k}^{(d)}$, $d=1,\ldots,D$.
\begin{remark}
При $D=1$ $\tX$ "--- одномерный временной ряд, и приведенный выше алгоритм совпадает с алгоритмом Basic SSA, описанный в~\cite{ssa_an}.
\end{remark}

\section{Toeplitz MSSA}\label{toeplitz}
В случае анализа стационарных рядов можно улучшить базовый метод, используя тёплицево разложение матрицы $\bfX$.
\begin{definition}
	Случайный процесс $\xi=(\xi_1,\ldots, \xi_n,\ldots)$ называется стационарным, если $\forall k\geqslant1$ $\mathsf E\xi_k=0$ и $\forall k,l\geqslant1$
	\[
	K(k, l)\overset{\text{def}}=\operatorname{cov}(\xi_k, \xi_l)= \widetilde{K}(k-l).
	\]
\end{definition}
\begin{definition}
	Детерминированный временной ряд $\tX=(x_1,\ldots,x_n,\ldots)$ называют стационарным, если существует функция $R_x(k)$ $(-\infty<k<+\infty)$ такая, что $\forall k,l\geqslant0$
	\begin{equation*}
		R_x^{(N)}(k,l)\overset{\text{def}}{=}\frac{1}{N}\sum_{m=1}^Nx_{k+m}x_{l+m}\underset{N\to\infty}\longrightarrow R_x(k - l).
	\end{equation*}
\end{definition}
\begin{definition}
	Пусть $\tX=\{\tX^{(d)}\}_{d=1}^D$ "--- $D$-канальный временной ряд. Ряд $\tX$ называют стационарным, если каждый канал $\tX^{(d)}$ "--- стационарный.
\end{definition}

\begin{definition}
	Пусть $\tX=\{\tX^{(d)}\}_{d=1}^D$ "--- $D$-канальный временной ряд с $N_d=N$. Зафиксируем $1<M<N$. Матрица $\bfT_{l,k}^{(M)}\in \mathbb{R}^{M\times M}$ c элементами
	\begin{equation*}\label{eq:block_elements}
		\left(\bfT^{(M)}_{l,k}\right)_{ij}=\frac{1}{N-|i-j|}\sum_{n=1}^{N-|i-j|} x^{(l)}_nx^{(k)}_{n+|i-j|},\ 1\leqslant i,j\leqslant M,
	\end{equation*}
	является оценкой ковариационной матрицы $l$ и $k$-го каналов.
\end{definition}

Toeplitz MSSA отличается от MSSA только другим разложением $\bfX$ в сумму матриц ранга $1$~\eqref{eq:svd}. В работе~\cite{Plaut1994SpellsOL} предложен один способ разложения $\bfX$, который мы назовем Block. Вместе с ним рассмотрим другой вариант разложения, который назовем Sum. 
\subsection{Toeplitz Block MSSA}
Рассмотрим блочную матрицу $$\bfT_{\text{Block}}=\begin{pmatrix}
	\bfT^{(K)}_{1,1} & \bfT^{(K)}_{1,2} & \cdots & \bfT^{(K)}_{1,D} \\
	\bfT^{(K)}_{2,1} & \bfT^{(K)}_{2,2} & \cdots & \bfT^{(K)}_{2,D} \\
	\vdots           & \vdots           & \ddots & \vdots           \\
	\bfT^{(K)}_{D,1} & \bfT^{(K)}_{D,D} & \cdots & \bfT^{(K)}_{D,D}
\end{pmatrix} \in \mathbb{R}^{DK\times DK},$$ где $K = N - L + 1$. Найдя ортонормированные собственные векторы $Q_1,\ldots,Q_{DK}$ матрицы $\bfT$, получаем разложение траекторной матрицы $\bfX$:
\begin{equation}\label{eq:block_decomposition}
	\mathbf{X}=\sum_{i=1}^{DK} \sigma_i P_i Q_i^\rmT=\bfX_1+\ldots+\bfX_{DK},
\end{equation}
где $Z_i=\bfX Q_i$, $P_i=Z_i/\|Z_i\|$, $\sigma_i=\|Z_i\|$.
\subsection{Toeplitz Sum MSSA}
Рассмотрим матрицу $\bfT_{\text{Sum}}=\sum_{d=1}^D \bfT_{d,d}^{(L)}\in\mathbb{R}^{L\times L}$. Найдем ортонормированные собственные векторы $P_1,\ldots,P_L$ матрицы $\bfT_{\text{Sum}}$ и разложим траекторную матрицу $\traj$ следующим образом:
\begin{equation}\label{eq:sum_decomposition}
	\mathbf{X}=\sum_{i=1}^L\sigma_i P_iQ_i^\rmT=\bfX_1+\ldots+\bfX_L,
\end{equation}
где $S_i=\bfX^\rmT P_i$, $Q_i=S_i/\|S_i\|$, $\sigma_i=\|S_i\|$.
\begin{remark}
Toeplitz Sum MSSA можно использовать для временных рядов с разными длинами каналов, в отличие от Toeplitz Block MSSA.
\end{remark}
\begin{remark}
	Пусть $\bfX=\sum_i \sigma_i P_i Q_i^T$ "--- любое из разложений \eqref{eq:svd}, \eqref{eq:block_decomposition} или \eqref{eq:sum_decomposition}. Будем называть $P_i$ \emph{левыми}, а $Q_i$ "--- \emph{правыми векторами} матрицы $\bfX$.
\end{remark}
\section{Численное сравнение методов}
Посмотрим на точность базового и модифицированных методов MSSA для разных значений параметра $L$. Рассмотрим следующий двухканальный временной ряд длины ${N=71}$:
$$\{\tF^{(1)}, \tF^{(2)}\}=\{\tS^{(1)},\tS^{(2)}\} + \{\tN^{(1)},\tN^{(2)}\},$$
где $\tS^{(1)}$, $\tS^{(2)}$ "--- некоторые сигналы, а $\tN^{(1)}$, $\tN^{(2)}$ "--- независимые реализации гауссовского белого шума с $\sigma=5$. Рассмотрим 3 случая, первые два из которых рассматривались ранее в~\cite{Golyandina_2015}:

\begin{enumerate}
	\item Косинусы с одинаковыми частотами:
	      \[
		      s_n^{(1)}=30\cos(2\pi n/12),\quad s_n^{(2)}=20\cos(2\pi n/12),\quad n=1,\ldots, N.
	      \]
	\item Косинусы с разными частотами:
	      \[
		      s_n^{(1)}=30\cos(2\pi n/12),\quad s_n^{(2)}=20\cos(2\pi n/8),\quad n=1,\ldots, N.
	      \]
	\item Полиномы первой степени (нестационарные ряды):
	      \[
		      s_n^{(1)}=1.2n,\quad s_n^{(2)}=0.8n,\quad n=1,\ldots,N.
	      \]
\end{enumerate}
В качестве оценки точности восстановления сигнала было взято среднеквадратичное отклонение от истинного значения. В таблице~\ref{tab:mse} представлены результаты на основе $10000$ реализаций шума. Наиболее точные результаты для каждого метода были выделены жирным шрифтом. Лучший результат для каждого случая выделен отдельно синим.
\begin{table}[h]
	\centering
	\caption{MSE восстановления сигнала.}
		\begin{tabular}{cccccc}\hline
			Случай 1 ($\omega_1=\omega_2$)   & $L=12$  & $L=24$                     & $L=36$                     & $L=48$                     & $L=60$          \\
			\hline
			SSA                              & $3.25$  & $\mathbf{2.01}$            & $\mathbf{2.00}$            & $\mathbf{2.01}$            & $3.25$          \\
			\hline
			Toeplitz SSA                     & $3.2$   & $1.87$                     & $1.63$                     & $\mathbf{1.59}$            & $1.67$          \\
			\hline
			MSSA                             & $3.18$  & $1.83$                     & $1.59$                     & $\mathbf{1.47}$            & $2.00$          \\
			\hline
			Toeplitz Sum MSSA                & $3.17$  & $1.75$                     & $1.44$                     & $\mathbf{1.32}$            & $\mathbf{1.33}$ \\
			\hline
			Toeplitz Block MSSA              & $1.39$  & \textcolor{blue}{${\mathbf{1.26}}$} & \textcolor{blue}{${\mathbf{1.25}}$} & $1.33$                     & $1.97$          \\
			\hhline{======}
			Случай 2 ($\omega_1\ne\omega_2$) & $L=12$  & $L=24$                     & $L=36$                     & $L=48$                     & $L=60$          \\
			\hline
			SSA                              & $3.25$  & $\mathbf{2.01}$            & $\mathbf{2.00}$            & $\mathbf{2.01}$            & $3.25$          \\
			\hline
			Toeplitz SSA                     & $3.2$   & $1.87$                     & $1.63$                     & \textcolor{blue}{${\mathbf{1.59}}$} & $1.67$          \\
			\hline
			MSSA                             & $6.91$  & $3.77$                     & $3.07$                     & $\mathbf{2.88}$            & $3.84$          \\
			\hline
			Toeplitz Sum MSSA                & $6.88$  & $3.65$                     & $2.64$                     & $2.37$                     & $\mathbf{2.27}$ \\
			\hline
			Toeplitz Block MSSA              & $4.47$  & $3.67$                     & $\mathbf{3.22}$            & $\mathbf{3.23}$            & $3.8$           \\
			\hhline{======}
			Случай 3 (тренд)                 & $L=12$  & $L=24$                     & $L=36$                     & $L=48$                     & $L=60$          \\
			\hline
			SSA                              & $3.65$  & $2.08$                     & $\mathbf{1.96}$            & $2.08$                     & $3.65$          \\
			\hline
			Toeplitz SSA                     & $3.33$  & $\mathbf{2.43}$            & $3.74$                     & $7.84$                     & $16.29$         \\
			\hline
			MSSA                             & $3.42$  & $1.94$                     & $1.63$                     & \textcolor{blue}{${\mathbf{1.57}}$} & $2.27$          \\
			\hline
			Toeplitz Sum MSSA                & $3.32$  & $\mathbf{2.24}$            & $3.04$                     & $5.91$                     & $11.95$         \\
			\hline
			Toeplitz Block MSSA              & $12.55$ & $6.18$                     & $2.97$                     & $\mathbf{1.78}$            & $1.97$          \\
			\hline
		\end{tabular}
	\label{tab:mse}
\end{table}

Как видно из таблицы~\ref{tab:mse}, в первом случае метод Block лучше всего выделял сигнал. В случае разных частот каналы имеют разную структуру, поэтому наиболее оптимальным является использовать Toeplitz SSA для каждого канала по отдельности.  В третьем случае мы имеем дело с нестационарными рядами одинаковой структуры, поэтому стандартный MSSA справляется лучше всего.

Заметим, что преимущество Block перед Sum в первом случае не очень больше. Также, если сравнивать методы по трудоемкости, для оптимальной длины окна метод Sum численно эффективнее Block: в случае Sum для оптимального $L\approx 2N/3$ строится матрица размера $2N/3\times2N/3$, в случае Block $L\approx N/2$ и матрица размера $DN/2\times DN/2$. И еще раз отметим, что Sum можно применять к временным рядам разной длины, поэтому рекомендуется использовать именно Toeplitz Sum MSSA.

\chapter{Метод Monte-Carlo MSSA}\label{chpt:mc-ssa}
\section{Постановка задачи}
Рассмотрим задачу поиска сигнала (не случайной составляющей) в многоканальном временном ряде. Модель выглядит следующим образом:
\[
\tX=\tS + \boldsymbol{\xi},
\]
где $\tS$ "--- сигнал, $\bfxi$ "--- какой-то шум. Тогда нулевая гипотеза $H_0:\tS=0$ (отсутствие сигнала, ряд состоит из чистого шума) и альтернатива $H_1:\tS\ne0$ (ряд содержит сигнал, например, периодическую составляющую). 
\begin{definition}
	Случайный вектор $\boldsymbol{\xi}=(\xi_1,\dots,\xi_N)$ называют красным шумом с параметрами $\varphi$ и $\delta$, если $\xi_n = \varphi\xi_{n-1} + \delta\varepsilon_n$, где $0<\varphi<1$, $\varepsilon_n$ — белый гауссовский шум со средним значением 0 и дисперсией 1 и $\xi_1$ имеет нормальное распределение с нулевым средним и дисперсией $\delta^2/(1-\varphi^2)$.
\end{definition}
%\begin{definition}
%	Пусть $\bfxi$ "--- красный шум с параметрами $\varphi$ и $\delta$. Теоретическая %корреляционная матрица красного шума $\bfxi$ "--- матрица с элементами $\varphi^{|i-j|}$.
%\end{definition}
В этой и следующей главах под шумом будем подразумевать именно красный, причем в данной главе с известными параметрами.  Также будем рассматривать только односторонние критерии.

\section{Одиночный тест}
Пусть $\bfxi$ "--- красный шум. Зафиксируем длину окна $L$ и обозначим траекторную матрицу ряда $\boldsymbol{\xi}$ как $\mathbf\Xi$. Рассмотрим вектор $W\in \R^{L}$ такой, что $\|W\|=1$. Введем величину
\[
	p=\|\mathbf{\Xi}\transponse W\|^2.
\]
Статистикой критерия является величина
\[
	\widehat{p}=\|\traj\transponse W\|^2.
\]
Если вектор $W$ "--- синусоида с частотой $\omega$, то $\widehat{p}$ отражает вклад частоты $\omega$ в исходный ряд.

Рассмотрим алгоритм статистического критерия проверки наличия сигнала в ряде с проекцией на один вектор $W$, описанный в работе~\cite{Golyandina_2023}.
\begin{algorithm}{Одиночный тест~\cite{Golyandina_2023}}
	\begin{enumerate}
		\item Построить статистику критерия $\widehat p$.
		\item Построить доверительную область случайной величины $p$: интервал от нуля до $(1-\alpha)$-квантиля, где $\alpha$ "--- уровень значимости.
		\item Если $\widehat p$	не попадает в построенный интервал "--- $H_0$ отвергается.
	\end{enumerate}
\end{algorithm}
Построенная доверительная область называется \textit{прогнозируемым интервалом} с уровнем доверия $1-\alpha$.
\begin{remark}
В большинстве случаев, распределение $p$ неизвестно. Поэтому оно оценивается методом Монте-Карло: берется $G$ реализаций случайной величины $\boldsymbol\xi$, для каждой вычисляется $p$ и строится эмпирическое распределение. В связи с этим описанный выше алгоритм называют методом Monte-Carlo SSA.
\end{remark}
\begin{remark}
Если частота $\omega$ сигнала $\tS$ известна, то в качестве $W$ можно взять синусоиду с частотой $\omega$. Но на практике $\omega$ редко бывает известна, что делает данный критерий несостоятельным против $H_1$.
\end{remark}
\section{Множественный тест}
%Пусть теперь частоты периодических компонент неизвестны (что не редкость на практике), но известен диапазон частот и нужно проверить, что в ряде присутствует сигнал с хотя бы одной частотой из заданного диапазона. Тогда нулевая гипотеза $H_0$ о том, что ряд не содержит сигнала ни на одной из частот из рассматриваемого диапазона, а альтернатива $H_1$ "--- ряд содержит сигнал с хотя бы одной частотой, принадлежащей рассматриваемому диапазону.
Пусть теперь частоты периодических компонент неизвестны, что не редкость на практике. Тогда подобно одиночному тесту рассмотрим набор $W_1,\ldots,W_H$ векторов для проекции, и для каждого $k=1,\ldots,H$ построим статистику критерия $\widehat p_k$:

%Пусть $W_1,\ldots,W_H$ "--- вектора для проекции. Для каждого $k=1,\ldots,H$ строится статистика критерия $\widehat p_k$:
\begin{equation}\label{eq:mc-ssa_statisctics}
	\widehat p_k = \|\bfX^\rmT W_k\|^2,\quad k=1,\ldots,H.
\end{equation}
В таком случае нужно построить $H$ предсказательных интервалов для каждого $W_k$ по выборкам $P_k=\{p_{ki}\}_{i=1}^G$ с элементами
\begin{equation}\label{eq:mc-ssa_h0}
	p_{ki}=\|\mathbf{\Xi}_i\transponse W_k\|^2,\quad i=1,\ldots,G;\ k=1,\ldots,H,
\end{equation}
где $G$ "--- количество суррогатных реализаций $\boldsymbol{\xi}$, $\mathbf{\Xi}_i$ "--- траекторная матрица $i$-й реализации $\boldsymbol{\xi}$. 

В работе~\cite{Golyandina_2023} подробно описана проблема множественного тестирования, когда вероятность ложного обнаружения периодической составляющей для одной из рассматриваемых частот (групповая ошибка I рода) неизвестна и значительно превышает заданный уровень значимости (частота ошибок одиночного теста), и ее решение. Приведем модифицированный алгоритм построения критерия в случае множественного тестирования, который будем использовать в дальнейшем.
\begin{algorithm}{Multiple MC-SSA~\cite{Golyandina_2023}}\label{alg:multiple_mc-ssa}
	\begin{enumerate}
		\item Для $k=1,\dots,H$ вычисляется статистика $\widehat{p}_k$, выборка $P_k=\{p_{ki}\}_{i=1}^G$, ее среднее $\mu_k$ и стандартное отклонение $\sigma_k$.
		\item Вычисляется $\mathbf{\eta}=(\eta_1,\dots,\eta_G)$, где
		      \[
			      \eta_i=\max_{1\leqslant k\leqslant H}(p_{ki}-\mu_k)/\sigma_k,\quad i=1,\dots,G.
		      \]
		\item Находится $q_k$ как выборочный $(1-\alpha)$-квантиль $\eta$, где $\alpha$ "--- уровень значимости.
		\item Нулевая гипотеза не отвергается, если
		      \[
			      \max_{1\leqslant k\leqslant H}(\widehat{p}_k-\mu_k)/\sigma_k<q.
		      \]
		\item Если $H_0$ отвергнута, вклад $W_k$ (и соответствующей частоты) значим, если $\widehat{p}_k$ превосходит $\mu_k+q\sigma_k$. Таким образом, $[0,\mu_k+q\sigma_k]$ считаются скорректированными интервалами прогнозирования.
	\end{enumerate}
\end{algorithm}

\section{Выбор векторов для проекции}
Отметим, что в SSA правые векторы матрицы $\bfX$ становятся левыми заменой $L$ на $N-L+1$, поэтому рассматривать по-отдельности левые и правые не нужно. Это не так в случае MSSA, который рассмотрен ниже.

В данной работе в качестве $W_1, \ldots,W_H$ берутся левые векторы матрицы $\bfX$. Такой способ выбора векторов для проекции самый распространенный, поскольку, если есть значимые векторы, можно восстановить сигнал с помощью SSA на их основе. Но этот вариант, вообще говоря, дает радикальный критерий.

\section{MC-MSSA: отличие от одномерного случая}
MC-SSA легко обобщается на многомерный случай: нужно просто заменить SSA на MSSA и генерировать красный шум с тем же количеством каналов, что и у исходного ряда.

Стоит отметить, что, в отличие от одномерного случая, левые и правые векторы матрицы отличаются по построению $\bfX$~\eqref{eq:embedding_mssa}, поэтому в MC-MSSA в качестве векторов для проекции рассмотрены и левые, и правые векторы. Если $W_1,\ldots,W_H$ "--- левые векторы матрицы $\bfX$, метод совпадает с алгоритмом~\ref{alg:multiple_mc-ssa}. Если рассматривать в качестве векторов для проекции правые векторы, то в формулах~\eqref{eq:mc-ssa_statisctics} и~\eqref{eq:mc-ssa_h0} нужно заменить $\bfX$ на $\bfX^\rmT$ и $\mathbf{\Xi}_i$ на $\mathbf{\Xi}_i^\rmT$ соответственно.

\section{Поправка неточных критериев}\label{sect:correction}
Приведем алгоритм поправки, преобразовывающий радикальные и консервативные критерии в точные. Зафиксируем уровень значимости $\alpha^*$, количество выборок $M_1$ для оценки $\alpha_I(\alpha)$ и их объем $N$.
\begin{algorithm}{Поправка уровня значимости по зависимости $\alpha_I(\alpha)$}~\cite{Larin_2022}\label{alg:correction}
	\begin{enumerate}
		\item Моделируется $M_1$ выборок объема $N$ при верной $H_0$.
		\item По моделированным данным строится зависимость ошибки первого рода от уровня значимости $\alpha_I(\alpha)$.
		\item Рассчитывается формальный уровень значимости: $\widetilde{\alpha}^*=\alpha_I^{-1}(\alpha^*)$. Критерий с таким уровнем значимости является асимптотически точным при $M_1\to\infty$.
	\end{enumerate}
\end{algorithm}

\begin{definition}
	ROC-кривая "--- это кривая, задаваемая параметрически
	\[
		\begin{cases}
			x=\alpha_I(\alpha) \\
			y=\beta(\alpha)
		\end{cases},\quad \alpha\in[0,1],
	\]
	где $\alpha_I(\alpha)$ "--- функция зависимости ошибки первого рода $\alpha_I$ от уровня значимости $\alpha$, $\beta(\alpha)$ "--- функция зависимости мощности $\beta$ от уровня значимости $\alpha$.
\end{definition}
С помощью ROC-кривых можно сравнивать по мощности неточные (в частности, радикальные) критерии. Отметим, что для точного критерия ROC-кривая совпадает с графиком мощности, так как $\alpha_I(\alpha)=\alpha$.
\section{Численное сравнение методов}\label{mc-ssa_numeric_comparison}
Алгоритм поправки радикальных критериев плохо работает для сильно радикальных критериев. Как было показано в~\cite[Приложение  Б.2.4]{Larin_2022}, метод MC-SSA с проекцией на левые (или правые) векторы SVD разложения матрицы $\bfX$~\eqref{eq:svd} дает очень радикальный критерий для больших значений длины окна $L$, что делает невозможным построение поправки.

Однако, в одномерном случае было установлено~\cite{Larin_2022}, что если вместо SVD разложения матрицы $\traj$ использовать тёплицево, то радикальность критерия уменьшается, и уже можно применить поправку. Установим, что будет в многомерном случае, если использовать модификации, описанные в главе~\ref{toeplitz}.

Пусть количество каналов равно двум, количество суррогатных реализаций красного шума $G=1000$. Для оценки ошибки первого рода, будем рассматривать красный шум $\bfxi$ с параметрами $\varphi=0.7$ и $\delta=1$, а для оценки мощности будет рассматривать временной ряд $\tX=\tS+\bfxi$, где $\tS$ "--- сигнал с элементами
\[
	s_n^{(1)}=s_n^{(2)}=\cos(2\pi\omega n),\quad n=1,\ldots, N,
\]
где $\omega=0.075$, $N=100$.

Построим графики ошибки первого рода и ROC-кривые для каждой длины окна $L=10$, $20$, $50$, $80$, $90$. Будем воспринимать ROC-кривую как график мощности критерия, к которому был применен алгоритм~\ref{alg:correction}.
\begin{figure}[h!]
	\captionsetup[subfigure]{justification=Centering}
	\begin{subfigure}[t]{0.45\textwidth}
		\centering
		\includegraphics[width=0.7\textwidth]{img/type1error_sum_ev.pdf}
		\caption{Ошибка первого рода (Sum).}
		\label{fig:sum_ev_a}
	\end{subfigure}\hspace{\fill}
	\begin{subfigure}[t]{0.45\textwidth}
		\centering
		\includegraphics[width=0.7\textwidth]{img/type1error_mssa_ev.pdf}
		\caption{Ошибка первого рода (базовый MSSA).}
	\end{subfigure}
	\bigskip
	\begin{subfigure}[t]{0.45\textwidth}
		\centering
		\includegraphics[width=0.7\textwidth]{img/roc_sum_ev.pdf}
		\caption{ROC-кривая (Sum).}
	\end{subfigure}\hspace{\fill}
	\begin{subfigure}[t]{0.45\textwidth}
		\centering
		\includegraphics[width=0.7\textwidth]{img/roc_mssa_ev.pdf}
		\caption{ROC-кривая (базовый MSSA).}
	\end{subfigure}
	\caption{Сравнение методов Sum и базового MSSA (проекция на левые векторы).}
	\label{fig:sum_ev}
\end{figure}
\begin{figure}[h!]
	\captionsetup[subfigure]{justification=Centering}
	\begin{subfigure}[t]{0.45\textwidth}
		\centering
		\includegraphics[width=0.7\textwidth]{img/type1error_sum_fa.pdf}
		\caption{Ошибка первого рода (Sum).}
		\label{fig:sum_fa_a}
	\end{subfigure}\hspace{\fill}
	\begin{subfigure}[t]{0.45\textwidth}
		\centering
		\includegraphics[width=0.7\textwidth]{img/type1error_mssa_fa.pdf}
		\caption{Ошибка первого рода (базовый MSSA).}
	\end{subfigure}
	\bigskip
	\begin{subfigure}[t]{0.45\textwidth}
		\centering
		\includegraphics[width=0.7\textwidth]{img/roc_sum_fa.pdf}
		\caption{ROC-кривая (Sum).}
	\end{subfigure}\hspace{\fill}
	\begin{subfigure}[t]{0.45\textwidth}
		\centering
		\includegraphics[width=0.7\textwidth]{img/roc_mssa_fa.pdf}
		\caption{ROC-кривая (базовый MSSA).}
	\end{subfigure}
	\caption{Сравнение методов Sum и базового MSSA (проекция на правые векторы).}
	\label{fig:sum_fa}
\end{figure}

На рис.~\ref{fig:sum_ev} и~\ref{fig:sum_fa} векторы для проекции были взяты из разложения~\eqref{eq:sum_decomposition}. На рис.~\ref{fig:sum_ev_a} видно, что при $L>20$ метод радикальный, а наибольшая мощность достигается при $L=90$. На рис.~\ref{fig:sum_fa_a} отчетливо заметно, что метод радикальный для всех $L$. Наибольшая мощность наблюдается при $L=90$, но отметим, что из-за слишком большой ошибки первого рода построить ROC-кривую на промежутке [0,3) для $L=50$ и на всем промежутке для $L=10$ и $L=20$ не получилось.
\begin{figure}[h!]
	\captionsetup[subfigure]{justification=Centering}
	\begin{subfigure}[t]{0.45\textwidth}
		\centering
		\includegraphics[width=0.7\textwidth]{img/type1error_block_ev.pdf}
		\caption{Ошибка первого рода (Block).}
		\label{fig:block_ev_a}
	\end{subfigure}\hspace{\fill}
	\begin{subfigure}[t]{0.45\textwidth}
		\centering
		\includegraphics[width=0.7\textwidth]{img/type1error_mssa_ev.pdf}
		\caption{Ошибка первого рода (базовый MSSA).}
	\end{subfigure}
	\bigskip
	\begin{subfigure}[t]{0.45\textwidth}
		\centering
		\includegraphics[width=0.7\textwidth]{img/roc_block_ev.pdf}
		\caption{ROC-кривая (Block).}
	\end{subfigure}\hspace{\fill}
	\begin{subfigure}[t]{0.45\textwidth}
		\centering
		\includegraphics[width=0.7\textwidth]{img/roc_mssa_ev.pdf}
		\caption{ROC-кривая (базовый MSSA).}
	\end{subfigure}
	\caption{Сравнение методов Block и базового MSSA (проекция на левые векторы).}
	\label{fig:block_ev}
\end{figure}
\begin{figure}[h!]
	\captionsetup[subfigure]{justification=Centering}
	\begin{subfigure}[t]{0.45\textwidth}
		\centering
		\includegraphics[width=0.7\textwidth]{img/type1error_block_fa.pdf}
		\caption{Ошибка первого рода (Block).}
		\label{fig:block_fa_a}
	\end{subfigure}\hspace{\fill}
	\begin{subfigure}[t]{0.45\textwidth}
		\centering
		\includegraphics[width=0.7\textwidth]{img/type1error_mssa_fa.pdf}
		\caption{Ошибка первого рода (базовый MSSA).}
	\end{subfigure}
	\bigskip
	\begin{subfigure}[t]{0.45\textwidth}
		\centering
		\includegraphics[width=0.7\textwidth]{img/roc_block_fa.pdf}
		\caption{ROC-кривая (Block).}
	\end{subfigure}\hspace{\fill}
	\begin{subfigure}[t]{0.45\textwidth}
		\centering
		\includegraphics[width=0.7\textwidth]{img/roc_mssa_fa.pdf}
		\caption{ROC-кривая (базовый MSSA).}
	\end{subfigure}
	\caption{Сравнение методов Block и базового MSSA (проекция на правые векторы).}
	\label{fig:block_fa}
\end{figure}

На рис.~\ref{fig:block_ev} и~\ref{fig:block_fa} векторы для проекции были взяты из разложения~\eqref{eq:block_decomposition}. Если рассматривать проекцию на левые векторы, то на рис.~\ref{fig:block_ev_a} видно, что метод радикальный, а наибольшая мощность достигается при $L=20$. Проекция на правые векторы также дает радикальный критерий, как видно на рис.~\ref{fig:block_fa_a}. Наибольшая мощность наблюдается при $L=80$, но из-за слишком большой ошибки первого рода ROC-кривую для $L=10$ и $L=20$, для которых метод, предположительно, имеет б\'oльшую мощность, удалось построить не на всем промежутке.

\begin{table}[h]
	\caption{Результаты численного сравнения методов для оптимальных длин окна}
	\label{tab:res_mc-ssa}
	\centering
	\begin{tabular}{|c>{\centering\arraybackslash}m{1in}c>{\centering\arraybackslash}m{1in} >{\centering\arraybackslash}m{1in}cc|}\hline
		Метод & левые/правые векторы & $L$ & длина векторов & количество векторов & $\alpha_I(\alpha)$ AUC & ROC AUC \\
		\hline
		MSSA & левые & $50$* & $50$ & $50$ & $0.7455$ & $0.405$\\
		\hline
		MSSA & правые & $80$* & $42$ & $42$ & $0.849$ & $0.3954$\\
		\hline
		Block & левые & $20$* & $162$ & $20$ & $0.5823$ & $0.4326$ \\
		\hline
		Block & правые & $80$* & $80$ & $42$ & $0.8328$ & $0.3982$\\
		\hline
		Sum & левые & $90$ & $90$ & $22$ & $0.8185$ & $0.4402$ \\
		\hline
		Sum & правые & $90$* & $22$ & $22$ & $0.6441$ & $0.4415$ \\
		\hline
	\end{tabular}

\end{table}

В таблице~\ref{tab:res_mc-ssa} для каждого метода указана оптимальная длина окна (для которой удалось простроить ROC-кривую, звездочкой помечены $L$, которые могут не являться оптимальными), длина векторов для проекции, их количество, площадь под кривой ошибки первого рода, а также площадь под ROC-кривой при $\alpha_I\in[0,0.5]$. Видно, что MC-MSSA с проекцией на левые или правые векторы обеих модификаций мощнее, чем с проекцией на векторы базового MSSA.
\section{Выводы}
Подведем итоги. На данный момент для метода Sum оптимальной длиной окна является $L=90$, если рассматривать проекцию как на левые, так и на правые векторы. Для метода Block оптимальной длиной окна является $L=20$, если рассматривать проекцию на левые векторы, и $L=80$, если рассматривать проекцию на правые векторы.

Также все методы, кроме Sum, с проекцией на левые вектора сильно радикальные. Поэтому рекомендуется использовать вариант Sum с проекцией на левые векторы с $L=90$. 

\chapter{Метод Monte-Carlo SSA для реальных задач}\label{chpt:mc_ssa_real}
В главе~\ref{chpt:mc-ssa} мы предполагали, что параметры шума известны и нету мешающего сигнала (например, сезонности или тренда). В этой главе рассмотрим случаи, которые более близки к реальным задачам.

\section{Оценка параметров красного шума}
До сих пор мы предполагали, что параметры красного шума $\varphi$ и $\delta$ известны, но в реальных задачах редко возникает такая ситуация. В этой ситуации можно воспользоваться методом bootstrapping, который позволяет использовать оцененные параметры шума для построения критерия~\cite{Golyandina_2023}. Параметры красного шума оценивались функцией \textsf{arima} с параметром \textsf{method="CSS-ML"} из пакета \textsf{stats} на языке программирования \textsf{R}. 

В этом разделе рассмотрим следующую двухступенчатую оценку параметров: сначала оцениваются параметры на основе исходного ряда и применяется MC-SSA с поправкой, затем выделяется сигнал, если он обнаружится, и оцениваются параметры на основе <<остатка>>. Такая оценка точнее оценивает неизвестные параметры и увеличивает мощность MC-SSA, если верна $H_1$.

Проверим на практике, что такой подход даст результаты лучше, чем обычная оценка параметров без выделения сигнала. За альтернативу возьмем
\begin{equation}\label{eq:noise_est_h1}
s_n=A\cos(2\pi\omega n),\quad n=1,\ldots,N,
\end{equation}
с амплитудой $A=1.5$ и частотой $\omega\in(0,0.5)$. Параметры красного шума и длину ряда $N$ возьмем такими же, как в разделе~\ref{mc-ssa_numeric_comparison}. Для двухступенчатой оценки длина окна $\widetilde L=50$, $G=1000$, $\alpha=0.1$

Будем оценивать параметр $\varphi$, обозначим оценку за $\hat\varphi$. В качестве оценки точности было взято среднеквадратичное отклонение от истинного значения. В таблице~\ref{tab:param_estim} представлены результаты на основе 100 реализаций шума. Поскольку $\mathsf{MSE}\hat\varphi=\mathsf{D}\hat\varphi + \mathsf{bias}^2\hat\varphi$, в таблице также представлены значения дисперсии и смещения оценки. . 

\begin{table}[h]
	\centering
	\begin{tabular}{|cccccc|}
		\hline
		Обычная оценка & $\omega = 0.075$ & $\omega=0.175$ & $\omega=0.275$ & $\omega=0.375$ & $\omega=0.475$ \\
		\hline
		$\mathsf{MSE}\hat\varphi$ & $0.0053$ & $0.0124$ & $0.1185$ & $0.3109$ & $0.4018$ \\
		\hline
		$\mathsf{D}\hat\varphi$ & $0.0023$ & $0.0036$ & $0.007$ & $0.0189$ & $0.0204$ \\
		\hline
		$\mathsf{bias}\hat\varphi$ & $0.055$ & $-0.0938$ & $-0.3341$ & $-0.5406$ & $-0.6178$\\
		\hhline{======}
		Двухступенчатая оценка & $\omega = 0.075$ & $\omega=0.175$ & $\omega=0.275$ & $\omega=0.375$ & $\omega=0.475$ \\
		\hline
		$\mathsf{MSE}\hat\varphi$ & $0.0091$ & $0.0057$ & $0.0038$ & $0.0098$ & $0.0129$ \\
		\hline
		$\mathsf{D}\hat\varphi$ & $0.0084$ & $0.0056$ & $0.0037$ & $0.0085$ & $0.0101$ \\
		\hline
		$\mathsf{bias}\hat\varphi$ & $-0.0284$ & $-0.0119$ & $-0.0107$ & $-0.0375$ & $-0.0538$\\
		\hline
	\end{tabular}
	\caption{Оценка параметров красного шума}
	\label{tab:param_estim}
\end{table}

По таблице~\ref{tab:param_estim} видно, что основной вклад в ошибку оценки вносит смещение, описанная выше процедура это смещение сильно уменьшает, делая слабо-отрицательным.
\begin{remark}
	В таблице~\ref{tab:param_estim} были взяты такие $\omega$ с целью показать общий случай, поскольку если $\widetilde L\omega$ "--- целое, то SSA точнее выделяет сигнал~\cite{ssa_an} и, следовательно, сделает оценку параметров лучше.
\end{remark}

Теперь сравним графики ошибок первого рода и ROC-кривые критерия MC-SSA против альтернативы~\eqref{eq:noise_est_h1} с $\omega=0.075$. Длины окна $L$ будем брать те же, что и в разделе~\ref{mc-ssa_numeric_comparison}.

\begin{figure}[h!]
	\captionsetup[subfigure]{justification=Centering}
	\begin{subfigure}[t]{0.45\textwidth}
		\centering
		\includegraphics[width=0.85\textwidth]{img/type1error_arima.pdf}
		\caption{Ошибка первого рода (обычная оценка)}
		\label{fig:arima_type1error}
	\end{subfigure}\hspace{\fill}
	\begin{subfigure}[t]{0.45\textwidth}
		\centering
		\includegraphics[width=0.85\textwidth]{img/type1error_extract.pdf}
		\caption{Ошибка первого рода (двухступенчатая оценка)}
		\label{fig:extract_type1error}
	\end{subfigure}
	\bigskip
	\begin{subfigure}[t]{0.45\textwidth}
		\centering
		\includegraphics[width=0.85\textwidth]{img/roc_arima.pdf}
		\caption{ROC-кривая (обычная оценка)}
		\label{fig:arima_roc}
	\end{subfigure}\hspace{\fill}
	\begin{subfigure}[t]{0.45\textwidth}
		\centering
		\includegraphics[width=0.85\textwidth]{img/roc_extract.pdf}
		\caption{ROC-кривая (двухступенчатая оценка)}
		\label{fig:extract_roc}
	\end{subfigure}
	\caption{Сравнение обычной и двухступенчатой оценок}
\end{figure}

По рис.~\ref{fig:arima_type1error} и~\ref{fig:extract_type1error} видно, графики ошибок первого рода примерно одинаковые для всех длин окна, что естественно, поскольку сигнала нет, и, следовательно, оценки параметров приблизительно одинаковые. А если посмотреть на ROC-кривые на рис.~\ref{fig:arima_roc} и~\ref{fig:extract_roc}, заметно повышение мощности при $L=10$, $20$, $50$. Поскольку для оптимальной длины окна $L=90$ разницы в мощности нет, далее будем оценивать параметры шума обычным, не двухступенчатым, способом.

\section{Наличие мешающего сигнала}
Пусть извезтно, что во временном ряде присутствует некоторый сигнал, но, возможно, еще есть какой-то другой. Тогда модель выглядит следующим образом:
\[
\tX=\tF + \tS + \boldsymbol{\xi},
\]
где $\tF$ "--- мешающий сигнал, $\tS$ "--- неизвестный сигнал и $\boldsymbol{\xi}$ "--- красный шум.
Будем проверять следующую нулевую гипотезу с альтернативой:
\begin{align*}
&H_0: \tS=0,\\
&H_1: \tS = \{\cos(2\pi\omega n)\}_{n=1}^N,
\end{align*}
где $\omega=0.075$.
\begin{algorithm}{MC-SSA с мешающим сигналом}
\begin{enumerate}
	\item Находится приближенное значение мешающего сигнала $\hat{\tF}$ и оцениваются параметры $\boldsymbol{\xi}$ на основе остатка $\tilde\tX=\tX-\hat{\tF}$.
	\item Находятся левые векторы $P_1,\ldots,P_L$ траекторной матрицы временного ряда $\tilde\tX$, полученные из разложения~\eqref{eq:sum_decomposition}. 
	\item Применяется MC-SSA к исходному ряду $\tX$ с проекцией на векторы $P_1,\ldots,P_L$, при этом суррогатными рядами являются реализации случайной величины $\boldsymbol{\eta}$:
	\[
	\boldsymbol{\eta} = \boldsymbol{\xi} + \hat\tF.
	\]
\end{enumerate}
\end{algorithm}

\subsection{Периодическая компонента}\label{sect:periodic_case}
Рассмотрим в качестве мешающего сигнала синусоиду
\[
f_n=A\cos(2\pi\omega n),\quad n=1,\ldots,N,
\]
с амплитудой $A=3$ и частотой $\omega=0.25$.

Будем выделять периодическую компоненту при помощи SSA: будем оценивать доминирующую частоту левых векторов с помощью метода \textsf{ESPRIT}~\cite[Раздел 3.1]{SSA_R} и на шаге группировки (раздел~\ref{sect:grouping}) будем брать две компоненты с наиболее близкими к $\omega$ частотами. 
\begin{figure}[h!]
	\captionsetup[subfigure]{justification=Centering}
	\begin{subfigure}[t]{0.5\textwidth}
		\centering
		\includegraphics[width=\textwidth]{img/type1error_sin.pdf}
		\caption{Ошибка первого рода}
		\label{fig:sin_type1error}
	\end{subfigure}\hspace{\fill}
	\begin{subfigure}[t]{0.5\textwidth}
		\centering
		\includegraphics[width=\textwidth]{img/roc_sin.pdf}
		\caption{ROC-кривая}
		\label{fig:sin_roc}
	\end{subfigure}
	\begin{subfigure}[t]{0.5\textwidth}
		\centering
		\includegraphics[width=\textwidth]{img/type1error_sin_est_noise.pdf}
		\caption{Ошибка первого рода (оцененные параметры шума)}
		\label{fig:sin_est_noise_type1error}
	\end{subfigure}\hspace{\fill}
	\begin{subfigure}[t]{0.5\textwidth}
		\centering
		\includegraphics[width=\textwidth]{img/roc_sin_est_noise.pdf}
		\caption{ROC-кривая (оцененные параметры шума)}
		\label{fig:sin_est_noise_roc}
	\end{subfigure}
	\begin{subfigure}[t]{0.5\textwidth}
		\centering
		\includegraphics[width=\textwidth]{img/type1error_sin_est_noise_signal.pdf}
		\caption{Ошибка первого рода (оцененный мешающий сигнал и параметры шума)}
		\label{fig:sin_est_noise_signal_type1error}
	\end{subfigure}\hspace{\fill}
	\begin{subfigure}[t]{0.5\textwidth}
		\centering
		\includegraphics[width=\textwidth]{img/roc_sin_est_noise_signal.pdf}
		\caption{ROC-кривая (оцененный мешающий сигнал и параметры шума)}
		\label{fig:sin_est_noise_signal_roc}
	\end{subfigure}
	
	
	%\begin{subfigure}[t]{\textwidth}
	%	\centering
	%	\includegraphics[width=0.7\textwidth]{img/power_sin.pdf}
	%	\caption{Мощность}
	%	\label{fig:sin_power}
	%\end{subfigure}
	\caption{Анализ метода, когда мешающий сигнал "--- периодическая компонента}
	\label{fig:sin}
\end{figure}

На рис.~\ref{fig:sin} представлены графики ошибок первого рода и ROC-кривые следующих критериев: когда мешающий сигнал и параметры шума известны точно, когда $\tF$ известен точно, но параметры шума оцениваются, и когда и мешающий сигнал, и параметры шума оцениваются. Графики ошибок первого рода на рис.~\ref{fig:sin_type1error},~\ref{fig:sin_est_noise_type1error} и~\ref{fig:sin_est_noise_signal_type1error} похожи друг на друга, а отклонение от случая, когда все известно, можно объяснить погрешностью при оценке неизвестных параметров. После применения поправки из раздела~\ref{sect:correction} критерии становятся точными для любой длины окна и ROC-кривые на рис.~\ref{fig:sin_roc},~\ref{fig:sin_est_noise_roc} и~\ref{fig:sin_est_noise_signal_roc} представляют собой графики мощности этих критериев. Таким образом, наибольшая мощность во всех трех случаях достигается при $L=90$.
\begin{figure}[h!]
	\centering
	\includegraphics[width=0.8\textwidth]{img/roc_sin_copm.pdf}
	\caption{Сравнение мощности критериев (мешающий сигнал "--- периодическая компонента)}
	\label{fig:power_comp_sin}
\end{figure}

На рис.~\ref{fig:power_comp_sin} представлена ROC-кривая критериев для оптимального $L$. Как видно из графика, при оценке параметров шума/мешающего сигнала мощность падает, но незначительно (примерно на 10\%). Также отметим, что оценка мешающего сигнала никак не повлияла на мощность.  
%Будем смотреть на графики ошибки первого рода, мощности и ROC-кривые этого критерия для разных длин окна, как в разделе~\ref{mc-ssa_numeric_comparison}. Начнем с ситуации, когда сигнал и параметры шума известны точно, постепенно приходя к случаю, когда и сигнал, и параметры шума нужно оценивать. 


%На рис.~\ref{fig:sin} представлены графики ошибки первого рода, мощности и ROC-кривые критерия, когда сигнал и параметры шума известны точно. График ошибки первого рода на рис.~\ref{fig:sin_type1error} показывает, что при длине окна $L=10$ критерий консервативный, при $L=20$ "--- приблизительно точный, при остальных $L$ "--- радикальный. Таким образом, имеет смысл рассматривать мощность на рис.~\ref{fig:sin_power} только для $L=10,20$. ROC-кривая на рис.~\ref{fig:sin_roc} похожа на график мощности при $L=20$, что соответствует теории. После применения поправки наибольшая мощность достигается при длине окна $L=90$.

%\begin{figure}[h!]
%	\captionsetup[subfigure]{justification=Centering}
%	\begin{subfigure}[t]{\textwidth}
%		\centering
%		\includegraphics[width=0.7\textwidth]{img/type1error_sin_est_noise.pdf}
%		\caption{Ошибка первого рода}
%		\label{fig:sin_est_noise_type1error}
%	\end{subfigure}
%	\begin{subfigure}[t]{\textwidth}
%		\centering
%		\includegraphics[width=0.7\textwidth]{img/power_sin_est_noise.pdf}
%		\caption{Мощность}
%		\label{fig:sin_est_noise_power}
%	\end{subfigure}
%	\begin{subfigure}[t]{\textwidth}
%		\centering
%		\includegraphics[width=0.7\linewidth]{img/roc_sin_est_noise.pdf}
%		\caption{ROC-кривая}
%		\label{fig:sin_est_noise_roc}
%	\end{subfigure}
%	\caption{Анализ метода с оцененными параметрами шума (сигнал "--- периодическая компонента)}
%	\label{fig:sin_est_noise}
%\end{figure}

%На рис.~\ref{fig:sin_est_noise} представлены графики ошибки первого рода, мощности и ROC-кривые критерия, когда сигнал известен точно, но параметры шума оцениваются. График ошибки первого рода на рис.~\ref{fig:sin_est_noise_type1error} показывает, что при длине окна $L=10$ при $\alpha<0.82$ и $L=20$ при $\alpha<0.58$ критерий консервативный, в иных случаях --- радикальный. Таким образом, имеет смысл рассматривать мощность на рис.~\ref{fig:sin_est_noise_power} только для $L=10$ при $\alpha<0.82$ и $L=20$ при $\alpha<0.58$. Наибольшая мощность поправленного критерия достигается при длине окна $L=90$. Отклонение от случая с известным сигналом и параметрами шума можно объяснить погрешностями при оценке параметров.
%\begin{figure}[h!]
%	\captionsetup[subfigure]{justification=Centering}
%	\begin{subfigure}[t]{\textwidth}
%		\centering
%		\includegraphics[width=0.7\textwidth]{img/type1error_sin_est_noise_signal.pdf}
%		\caption{Ошибка первого рода}
%		\label{fig:sin_est_noise_signal_type1error}
%	\end{subfigure}
%	\begin{subfigure}[t]{\textwidth}
%		\centering
%		\includegraphics[width=0.7\textwidth]{img/power_sin_est_noise_signal.pdf}
%		\caption{Мощность}
%		\label{fig:sin_est_noise_signal_power}
%	\end{subfigure}
%	\begin{subfigure}[t]{\textwidth}
%		\centering
%		\includegraphics[width=0.7\linewidth]{img/roc_sin_est_noise_signal.pdf}
%		\caption{ROC-кривая}
%		\label{fig:sin_est_noise_signal_roc}
%	\end{subfigure}
%	\caption{Анализ метода с оцененным сигналом и параметрами шума (сигнал "--- периодическая компонента)}
%	\label{fig:sin_est_noise_signal}
%\end{figure}

%На рис.~\ref{fig:sin_est_noise_signal} представлены графики ошибки первого рода, мощности и ROC-кривые критерия, когда оценивается и сигнал, и параметры шума. Поскольку неопределенности стало еще больше, отклонение от случая с известным сигналом и параметрами шума стало еще сильнее. График ошибки первого рода на рис.~\ref{fig:sin_est_noise_signal_type1error} показывает, что при длине окна $L=10$ при $\alpha<0.67$ и $L=20$ при $\alpha<0.2$ критерий консервативный, в иных случаях --- радикальный. Наибольшая мощность поправленного критерия достигается при длине окна $L=90$, как и в предыдущих случаях.




\subsection{Тренд}
Отдельно рассмотрим вариант, когда мешающий сигнал "--- тренд, т.е. медленно меняющаяся компонента. Рассмотрим следующий экспоненциальный ряд:
\[
k_n=A e^{\alpha n},\quad n=1,\ldots,N,
\] 
где $A=0.2$, $\alpha=0.05$. 

Выделять тренд будем с помощью SSA: поскольку в SVD разложении~\eqref{eq:svd} сингулярные числа, соответствующие тренду, будут самыми большими среди всех сингулярных чисел, на шаге группировки (раздел~\ref{sect:grouping}) будем брать первые $r$ элементарных компонент, где $r$ "--- ранг тренда. В данном случае $r=1$.

\begin{figure}[h!]
	\captionsetup[subfigure]{justification=Centering}
	\begin{subfigure}[t]{0.5\textwidth}
		\centering
		\includegraphics[width=\textwidth]{img/type1error_trend.pdf}
		\caption{Ошибка первого рода}
		\label{fig:trend_type1error}
	\end{subfigure}\hspace{\fill}
	\begin{subfigure}[t]{0.5\textwidth}
		\centering
		\includegraphics[width=\textwidth]{img/roc_trend.pdf}
		\caption{ROC-кривая}
		\label{fig:trend_roc}
	\end{subfigure}
	\begin{subfigure}[t]{0.5\textwidth}
		\centering
		\includegraphics[width=\textwidth]{img/type1error_trend_est_noise.pdf}
		\caption{Ошибка первого рода (оцененные параметры шума)}
		\label{fig:trend_est_noise_type1error}
	\end{subfigure}\hspace{\fill}
	\begin{subfigure}[t]{0.5\textwidth}
		\centering
		\includegraphics[width=\textwidth]{img/roc_trend_est_noise.pdf}
		\caption{ROC-кривая (оцененные параметры шума)}
		\label{fig:trend_est_noise_roc}
	\end{subfigure}
	\begin{subfigure}[t]{0.5\textwidth}
		\centering
		\includegraphics[width=\textwidth]{img/type1error_trend_est_noise_signal.pdf}
		\caption{Ошибка первого рода (оцененный мешающий сигнал и параметры шума)}
		\label{fig:trend_est_noise_signal_type1error}
	\end{subfigure}\hspace{\fill}
	\begin{subfigure}[t]{0.5\textwidth}
		\centering
		\includegraphics[width=\textwidth]{img/roc_sin_est_noise_signal.pdf}
		\caption{ROC-кривая (оцененный мешающий сигнал и параметры шума)}
		\label{fig:trend_est_noise_signal_roc}
	\end{subfigure}
	\caption{Анализ метода, когда мешающий сигнал "--- тренд}
\label{fig:trend}
\end{figure}

На рис.~\ref{fig:trend} представлены графики ошибок первого рода и ROC-кривые следующих критериев: когда тренд и параметры шума $\varphi$ и $\delta$ известны точно, когда тренд известен точно, но параметры шума оцениваются, и когда и тренд, и параметры шума оцениваются. Как и в разделе~\ref{sect:periodic_case}, графики ошибок первого рода на рис.~\ref{fig:trend_type1error},~\ref{fig:trend_est_noise_type1error} и~\ref{fig:trend_est_noise_signal_type1error} сохраняют общую тенденцию при оценке параметров/тренда. По ROC-кривым на рис.~\ref{fig:trend_roc},~\ref{fig:trend_est_noise_roc} и~\ref{fig:sin_est_noise_signal_roc} видно, что оптимальной длиной окна является $L=90$. Стоит также отметить странное поведение мощности при $L=10$.

\begin{figure}[h!]
	\centering
	\includegraphics[width=0.8\textwidth]{img/roc_trend_copm.pdf}
	\caption{Сравнение мощности критериев (мешающий сигнал "--- тренд)}
	\label{fig:power_comp_trend}
\end{figure}

На рис.~\ref{fig:power_comp_trend} представлена ROC-кривая критериев для оптимального $L$. Аналогично случаю периодической компоненты, при оценке параметров шума/мешающего сигнала мощность падает незначительно. Также отметим, что критерий с оценкой тренда и параметров шума оказался немого мощнее, чем критерий с оценкой только параметров шума.  
%Будем смотреть на графики ошибки первого рода, мощности и ROC-кривые как в разделе~\ref{sect:periodic_case}. 
%\begin{figure}[h]
%	\captionsetup[subfigure]{justification=Centering}
%	\begin{subfigure}[t]{\textwidth}
%		\centering
%		\includegraphics[width=0.7\textwidth]{img/type1error_trend.pdf}
%		\caption{Ошибка первого рода}
%		\label{fig:trend_type1error}
%	\end{subfigure}
%	\begin{subfigure}[t]{\textwidth}
%		\centering
%		\includegraphics[width=0.7\textwidth]{img/power_trend.pdf}
%		\caption{Мощность}
%		\label{fig:trend_power}
%	\end{subfigure}
%	\begin{subfigure}[t]{\textwidth}
%		\centering
%		\includegraphics[width=0.7\linewidth]{img/roc_trend.pdf}
%		\caption{ROC-кривая}
%		\label{fig:trend_roc}
%	\end{subfigure}
%	\caption{Анализ метода, сигнал и параметры шума известны (сигнал "--- тренд)}
%	\label{fig:trend}
%\end{figure}
%
%На рис.~\ref{fig:trend} представлены графики ошибки первого рода, мощности и ROC-кривые критерия, когда сигнал и параметры шума известны точно. График ошибки первого рода на рис.~\ref{fig:trend_type1error} показывает, что при $L=10$ критерий консервативен, при $L=20$ "--- приблизительно точный (слабо радикальный), для остальных $L$ "--- радикальный. Поэтому график мощности на рис.~\ref{fig:trend_power} похож на ROC-кривую на рис~\ref{fig:trend_roc} при $L=20$. Наибольшая мощность достигается при длине окна $L=90$.
%
%\begin{figure}[h]
%	\captionsetup[subfigure]{justification=Centering}
%	\begin{subfigure}[t]{\textwidth}
%		\centering
%		\includegraphics[width=0.7\textwidth]{img/type1error_trend_est_noise.pdf}
%		\caption{Ошибка первого рода}
%		\label{fig:trend_est_noise_type1error}
%	\end{subfigure}
%	\begin{subfigure}[t]{\textwidth}
%		\centering
%		\includegraphics[width=0.7\textwidth]{img/power_trend_est_noise.pdf}
%		\caption{Мощность}
%		\label{fig:trend_est_noise_power}
%	\end{subfigure}
%	\begin{subfigure}[t]{\textwidth}
%		\centering
%		\includegraphics[width=0.7\linewidth]{img/roc_trend_est_noise.pdf}
%		\caption{ROC-кривая}
%		\label{fig:trend_est_noise_roc}
%	\end{subfigure}
%	\caption{Анализ метода с оцененными параметрами шума (сигнал "--- тренд)}
%	\label{fig:trend_est_noise}
%\end{figure}
%
%На рис.~\ref{fig:trend_est_noise} представлены графики ошибки первого рода, мощности и ROC-кривые критерия, когда сигнал известен точно, но параметры шума оцениваются. По графику ошибки первого рода на рис.~\ref{fig:trend_est_noise_type1error} видно, что при длине окна $L=10$ при $\alpha<0.7$ и $L=20$ при $\alpha<0.43$ критерий консервативный, в остальных случаях "--- радикальный. Как и в случае периодического сигнала, отклонение от предыдущего случая связано с погрешностями оценок параметров шума. По графику ROC-кривой на рис.~\ref{fig:trend_est_noise_roc} наибольшая мощность поправленного критерия наблюдается при длине окна $L=90$.
%
%\begin{figure}[h]
%	\captionsetup[subfigure]{justification=Centering}
%	\begin{subfigure}[t]{\textwidth}
%		\centering
%		\includegraphics[width=0.7\textwidth]{img/type1error_trend_est_noise_signal.pdf}
%		\caption{Ошибка первого рода}
%		\label{fig:trend_est_noise_signal_type1error}
%	\end{subfigure}
%	\begin{subfigure}[t]{\textwidth}
%		\centering
%		\includegraphics[width=0.7\textwidth]{img/power_trend_est_noise_signal.pdf}
%		\caption{Мощность}
%		\label{fig:trend_est_noise_signal_power}
%	\end{subfigure}
%	\begin{subfigure}[t]{\textwidth}
%		\centering
%		\includegraphics[width=0.7\linewidth]{img/roc_trend_est_noise_signal.pdf}
%		\caption{ROC-кривая}
%		\label{fig:trend_est_noise_signal_roc}
%	\end{subfigure}
%	\caption{Анализ метода с оцененным сигналом и параметрами шума (сигнал "---тренд)}
%	\label{fig:trend_est_noise_signal}
%\end{figure}
%
%На рис.~\ref{fig:trend_est_noise_signal} представлены графики ошибки первого рода, мощности и ROC-кривые критерия, когда сигнал и параметры шума неизвестны, и их нужно оценивать. По сравнению с случаем, когда оценивались только параметры шума, критерий стал еще радикальней, это видно на рис.~\ref{fig:trend_est_noise_signal_type1error}. Смотреть на график мощности на рис.~\ref{fig:trend_est_noise_signal_power} можно только при длине окна $L=10$ при $\alpha<0.77$ и $L=20$ при $\alpha<0.3$. Если воспользоваться поправкой, критерий становится точным для любой длины окна и ROC-кривая на рис.~\ref{fig:trend_est_noise_signal_roc} представляет собой мощность такого критерия. Таким образом, наибольшая мощность достигается при длине окна $L=90$.

\conclusion
В ходе данной работы для реализации двух методов Тёплицева MSSA был использован язык программирования $\tR$. Было получено, что в точности восстановления сигнала оба метода в большинстве случаев показывают лучший результат, чем обычный MSSA. Но в Monte-Carlo SSA метод Sum более предпочтителен, чем метод Block, что важно ввиду его простоты в реализации и структуры, подходящей под пакет Rssa~\cite{Rssa}.

Также было рассмотрено два способа оценки неизвестных параметров красного шума: обычным bootstrap'ом и двухступенчатой оценкой, описанной в данной работе. Было получено, что двухступенчатая оценка точнее оценивает истинные параметры красного шума, если, помимо красного шума, во временном ряде присутствует сигнал.

Еще были разобраны два примера Monte-Carlo SSA с мешающим сигналом. Были рассмотрены случаи, когда мешающий сигнал и параметры красного шума известны, когда оценивались только параметры красного шума, и когда оценивались и мешающий сигнал, и параметры красного шума.

В дальнейшем предполагается расширить набор примеров мешающих сигналов и выработать более общие рекомендации, а также рассмотрение алгоритмов в случае многомерных временных рядов.
\bibliographystyle{ugost2008}
\bibliography{report}
\end{document}